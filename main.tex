\documentclass{article}

% Language setting
% Replace `english' with e.g., `spanish' to change the document language
\usepackage[english]{babel}

% Set page size and margins
% Replace `letterpaper' with`a4paper' for UK/EU standard size
\usepackage[letterpaper,top=2cm,bottom=2cm,left=3cm,right=3cm,marginparwidth=1.75cm]{geometry}

% Useful packages
\usepackage{amsmath}
\usepackage{graphicx}
\usepackage[colorlinks=true, allcolors=blue]{hyperref}

\title{Euro-Par 2025 Workshop Proposal: Asynchronous Many-Task systems for Exascale 2025 (AMTE 2025)
}
\author{Patrick Diehl, Claudia Fohry, Steven R. Brandt, and Parsa Amini }

\begin{document}
\maketitle

%\begin{abstract}
%Your abstract.
%\end{abstract}å
\section*{Workshop Title}
Asynchronous Many-Task systems for Exascale 2025
\section*{Acronym}
AMTE 2025

\section*{Preference for length}
We prefer a full day, could be half-day

\section*{Organizers} 
\begin{itemize}
\item Patrick Diehl \\
Patrick Diehl is a research scientist at Los Alamos National Laboratory and an adjunct assistant professor at the Department for Physics and Astronomy. Prior to joining Los Alamos National Laboratory, he was a member of the STE$\vert\vert$AR group and a researcher at the Center for Computation and Technology at Louisiana State University. Before joining LSU, he was a postdoctoral fellow at the Laboratory for Multiscale Mechanics at Polytechnique Montreal. Patrick received a diploma in Computer Science from the University of Stuttgart and a Ph.D. in Applied Mathematics from the University of Bonn. His primary research interests are 1) Computational engineering with a focus on Peridynamic material models for application in solids, such as glassy or composite materials, and 2) High-performance computing, especially the asynchronous many-task system (AMT), e.g., the C++ standard library for parallelism and concurrency (HPX) for large heterogeneous computations.
\item Claudia Fohry\\
Claudia Fohry is an associate professor of Computer Science at
University of Kassel, Germany, where she leads the research group on
Programming Languages/Methodologies. She received a Diploma in
Mathematics in 1990 and a Ph.D. in Computer Science in
1992, both from Humboldt-Universität zu Berlin, Germany. In 2003, she
habilitated in Computer Science at Friedrich-Schiller-Universität
Jena, Germany. Her main research interests are parallel programming and
parallel algorithms. In recent years, she focused on task-based parallel
programming, runtime systems, and resilience.
\item Parsa Amini \\
Parsa Amini is a member of the STE$\vert\vert$AR group. His research interests include data addressing and relocation in high-performance computing applications, active global address spaces, and high-productivity parallel C++ codes. He has been involved with developing the HPX runtime system and HPX applications, such as the Phylanx distributed array toolkit and the Octo-Tiger astrophysics code. Parsa holds a Ph.D. degree in Computer Science from Louisiana State University.
\end{itemize}

\section*{Tentative program committee} 

\begin{itemize}
    \item Thomas Heller, Exasol, Germany
    \item Hartmut Kaiser, Louisiana State University, USA 
    \item Dirk Pleiter, KTH Royal Institute of Technology in Stockholm
    \item Roman Iakymchuk, Umea University, Sweden
    \item Erwin Laure, Max Planck Computing \& Data Facility, Germany
    \item Laxmikant (Sanjay) V. Kale, University of Illinois at Urbana-Champaign, USA
    \item Andrew Lumsdaine, Northwest Institute for Advanced Computing, USA
    \item Patricia Grubel, Los Alamos National Laboratory, USA
    \item Vassilios Dimakopoulos, University of Ioannina, Greece
    \item Metin H. Aktulga, Michigan State University, USA
    \item Galen Shipman, Los Alamos National Laboratory, USA
    \item Brad Richardson, Sourcery Institute, USA
    \item Kevin Huck, University of Oregon, USA
    \item Jeff Hammond, NVIDIA, USA
    \item Peter Thoman, University of Innsbruck, Austria
    \item Keita Teranishi, Sandia National Laboratory, USA 
    \item Pat McCormick, Los Alamos National Laboratory, USA
    \item Daisy Hollman, Google, USA
    \item Thomas Fahringer, University of Innsbruck, Austria
    \item Pedro Valero Lara, Oak Ridge National Laboratory, USA 
    \item Adrian Lemoine, AMD, USA
    \item Mikael Simberg, Swiss National Supercomputing Centre, Switzerland
    \item Joseph Schuchart, Stony Brook University\, USA
    \item Qinglei Cao, Saint Louis University
%C: to get more women, perhaps add:
%C: Karla Morris, Sandia, USA
%C: Michelle Strout, HP, USA
%C: (I'm rather skeptic that they would accept, but it looks better)
%C: perhaps delete 2-3 USA members, to reduce the impression of US focus 
\end{itemize}


\section*{Motivation of the workshop}

\subsection*{Scientific objective}

Supercomputers have begun to operate at exascale speed, and a tremendous amount of work has been invested in identifying and overcoming the challenges leading up to this moment. These challenges include load-balancing, fast data transfers, and efficient resource utilization.  Asynchronous Many-Task (AMT) models and runtime systems have shown that it is possible to address these challenges by providing additional mechanisms such as oversubscription, task/data locality, shared memory, and data dependence-driven execution. 
This workshop explores the advantages of AMT programming on modern and future  HPC systems. It will gather developers, users, and proponents of these models and systems to share experiences, discuss how they meet the challenges posed by today's heterogeneous Exascale system architectures, and explore opportunities for increased performance, robustness, and full-system utilization.

\subsection*{Interest to the Euro-Par community}

Emerging AMT programming models and runtime systems shield the programmer from the challenging task of identifying and managing parallelism by delegating this task to the runtime system. Although dozens of AMT programming systems for parallel and high-performance computing (HPC) exist today and are actively researched, they are not commonly used in HPC applications and algorithm implementations. It has been shown that AMT programming is beneficial for certain kinds of applications, and the workshop aims to increase awareness of these benefits to the HPC community. 
It will accomplish this, in part, by sharing experiences with these systems for different kinds of problems/applications running on different types of hardware. Another goal of this workshop is to gather AMT runtime experts and developers worldwide to create a stronger community. 
We will document the panel discussion and make the results available on the workshop’s web page. In addition, we intend to publish a workshop report.


\subsection*{Positioning with respect to the currently existing Euro-Par workshops}

To our knowledge, this is the only workshop that focuses on Asynchronous Many-Task programming systems.

\section*{Workshop Description}

\subsection*{Content}

The workshop will focus on the following areas:
\begin{itemize}
\item Novel AMT runtime environments
\item Experiences in using AMT runtimes
%\item Environments for large applications
\item Comparisons between AMT runtime environments
\item Mechanisms for task coordination (e.g. dataflow, fork-join)
\item Using AMT runtimes on accelerators or heterogeneous architectures.
%\item Experiences gathered from porting one large-scale parallel solution to another, e.g., MPI to %Charm++, etc.
%\item Profiling and performance monitoring of task-based environments
\item Benchmarks for AMT runtimes
\item Profiling, performance monitoring, and debugging tools for AMT environments
%\item Tools for debugging programs using task-based runtimes
\item Challenges to AMT runtimes in scaling to large clusters
\item Hardware challenges and solutions in using AMT environments
\item Task-based algorithms.
\end{itemize}

\subsection*{Format}

It will be a full-day or half-day workshop with one keynote speaker, one invited talk (if full-day), and a panel (if full-day).
The tentative schedule for the full-day workshop is:
\begin{itemize}
%C: To my mind, the morning/afternoon breaks are too short. 
%C: I would prefer 25-minute talks and longer breaks.
\item 9:00 am - 9:05 am, Opening Remarks
\item 9:05 am - 10:05 am, Keynote Talk  
\item 10:05 am - 10:20 am, Morning Break
\item 10:20 am - 10:50 am Selected paper talk
\item 10:50 am - 11:20 am, Selected paper talk
\item 11:20 am - 11:50 am, Selected paper talk
\item 11:50 am - 12:20 pm, Selected paper talk
\item 12:20 pm - 1:30 pm, Lunch Break 
\item 1:30 pm - 2:30 pm, Invited talk 
\item 2:30 pm - 3:00 pm, Selected paper talk
\item 3:00 pm - 3:30 pm, Selected paper talk
\item 3:30 pm -3:45 pm Afternoon break
\item 3:45 pm - 4:45 pm, Panel 
\end{itemize}

\subsection*{Organizational aspects}
We will advertise the workshop on appropriate mailing lists, e.g., hpc-announce, SIAM CSE, ACM, Grace Hopper, Women in HPC, Women in AI, and Gesellschaft für Informatik. 
%C: I'm not aware of a "Gesellschaft für Informatik" mailing list, and neither am I
%C: of any comparable list in Germany. Possibly we may advertise through user mailing lists of 
%C: supercomputing centers, but I would not insert them here.
Next, we will promote the workshop on social media and ask our home institutions’ communication and outreach departments for support. In addition, the organizers will compile a list of potential speakers and send personalized invitations to them. 

Workshop organizers value equality and diversity. We will strongly encourage women and BAME to submit papers to this workshop and will make sure to meet the following criteria when choosing speakers and committee members:
\begin{itemize}
\item Equity: we will make sure that
%C: "make sure" sounds pretty strong, especially as we can hardly reduce hotel costs and things like %C: that. Perhaps "strive to ensure", "do what we can to ensure" or something like this?
there are no barriers, biases, and obstacles that impede equal access and opportunity to submit papers to the workshop;
\item Diversity: 
\begin{itemize}
\item Committee diversity – we will recruit qualified committee members and reviewers such that the resulting populations are diverse concerning gender, home institutions, geography, and research expertise. In addition, we will try to have a mixed committee containing academic and industrial members. 
\item Speaker diversity – we will recruit invited speakers and panelists from diverse populations, backgrounds, and research areas to increase diversity in thinking and perspective. For those, we will advertise this workshop in groups such as “Women in HPC,” Women in AI, and the Grace Hopper conference mailing list.
\end{itemize}
\item Inclusion: we will create a workshop program and environment that is free from discrimination and where every participant feels welcome, included, respected, and safe.
\end{itemize}

All the papers submitted to the workshop will be peer-reviewed  (at least three committee members will review each paper). At the end of the workshop, we will publish its proceedings.

\subsection*{Pre-Workshop Timeline and Procedures}
\begin{itemize}
    \item Workshop notification: February 24, \the\year{}
    \item Initial announcing of the workshop: February 28, \the\year{}
    \item Workshop website development: March 7, \the\year{}
    \item Launch call for workshop papers: March 14, \the\year{}
    \item Deadlines for submission: May 6, \the\year{}
    \item Organizer meeting: May 7, \the\year{}
    \item Reviews complete: June 11, \the\year{}
    \item Reviewer committee meeting: June 12, \the\year{}
    \item Notification of acceptance: June 14, \the\year{}
    \item Camera-ready papers due: July 1, \the\year{}
    \item Program announcement: July 29, \the\year{}
    \item Workshop date: August 25 or 26, \the\year{}
    \item After the workshop website update: September 23, \the\year{}
    \item Workshop management report: September 16, \the\year{}
\end{itemize}

\subsection*{Workshop background}
This is the fourth proposal for the AMTE workshop as a part of the Euro-Par conference. The first workshop was held in 2021 as a full-day workshop, and five papers were accepted. The workshop included one keynote and one panel. The second workshop was part of the Euro-Par 2022 conference and was held as a half-day workshop. We had three papers, a keynote, and an invited talk. The third workshop was part of the Euro-Par 2023 conference and was held as a half-day workshop. We had four papers, a keynote, and an invited talk. The Task-Based Algorithms and Applications (TBAA 2020) panel was accepted at Supercomputing 2020, and its proceedings are published at \url{https://www.osti.gov/biblio/1764191/}. We intend to publish a similar workshop report for this event as well. In February 2023, the Workshop on Asynchronous Many-Task Systems and Applications was held at LSU. We received 15 talks, and we had three keynotes.

\subsection*{Link to international projects/initiatives}
\begin{itemize}
    \item https://gitlab.inria.fr/openmp/libkomp
    \item https://juliacomputing.com/
    \item https://legion.stanford.edu/
    \item https://stellar-group.org/
    \item https://github.com/STEllAR-GROUP/hpx
    \item https://github.com/STEllAR-GROUP/octotiger
    \item https://github.com/STEllAR-GROUP/phylanx
    \item https://charmplusplus.org/
    \item http://charm.cs.illinois.edu/research/charm
    \item https://www.ks.uiuc.edu/Research/namd/
    \item https://github.com/kokkos/kokkos
    \item https://github.com/NVIDIA/libcudacxx
\end{itemize}


\end{document}
